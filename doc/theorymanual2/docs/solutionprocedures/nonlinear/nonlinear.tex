\documentclass[11pt]{article}
\usepackage[english]{babel}
%\usepackage{fancyhdr, blindtext}
%\usepackage{lastpage}
\usepackage{graphicx}
\usepackage{ifpdf}
\usepackage{fancyhdr}
\usepackage[utf8]{inputenc}
\usepackage{amssymb,amsmath}
\oddsidemargin -10mm
\evensidemargin -10mm
\textwidth 180mm
\parindent 0mm

\def\mytitle{NonLinear Solution Procedures}
\def\myauthor{Borek Patzak}
\def\mydate{11/09/2019}

\usepackage{fancyhdr, blindtext}
\usepackage{lastpage}

\pagestyle{fancyplain}

\fancyhead{}
\fancyfoot{}
\fancyhead[L]{\bf \mytitle}
\fancyhead[R]{\large OOFEM.ORG}
\fancyfoot[L]{Responsible: \myauthor}
\fancyfoot[R]{Date: \mydate\ Page:\ \thepage/\pageref{LastPage} }
\renewcommand{\headrulewidth}{0.4pt}
\renewcommand{\footrulewidth}{0.4pt}
\setlength\headheight{14pt}

\fancypagestyle{titlestyle}{%
    \fancyhead[L]{}
    \fancyhead[R]{\large OOFEM.ORG}
    \lfoot{Responsible:~\myauthor}
    \renewcommand{\headrulewidth}{0.4pt}
    \renewcommand{\footrulewidth}{0.4pt}
}
 
%Redefine chapter by adding special first chapter page page-style
%\makeatletter
%    \let\stdchapter\chapter
%    \renewcommand*\chapter{%
%    \@ifstar{\starchapter}{\@dblarg\nostarchapter}}
%   \newcommand*\starchapter[1]{\stdchapter*{#1}\thispagestyle{chapterstyle}}
%    \def\nostarchapter[#1]#2{\stdchapter[{#1}]{#2}\thispagestyle{chapterstyle}}
%\makeatother

\renewcommand{\maketitle}[1]{
    %\begin{titlepage}
        %{
        \thispagestyle{titlestyle}
        \vspace*{15em}
        \noindent{\Huge \mytitle}\\
        \noindent\rule{\textwidth}{0.4pt}\\[2em]
        {\bf Summary}\\[0.5em]
        #1
        \newpage
        %}
    %\end{titlepage}
}


\newcommand{\oofem}{\htmladdnormallink{OOFEM}{http://www.oofem.org}\ }
\newcommand{\oofemlnk}[1]{\htmladdnormallink{#1}{http://www.oofem.org}\ }
\newcommand{\bp}{\htmladdnormallink{Bo\v{r}ek Patz\'{a}k}{http://mech.fsv.cvut.cz/~bp/bp.html}}


\begin{document}
\maketitle{This document contains introduction to nonlinear solution procedures.}

\section{Solution procedures for nonlinear systems}
We illustrate this on problem of nonlinear mechanics. Our starting point is general form of equilibrium equations expressing the balance between internal $\mbf{f}^{int}$ and external $\mbf{f}^{ext}$
$$
\mbf{f}^{int}(\mbf{r}) = \mbf{f}^{ext}
$$
Suppose we are looking for an equilibrium at the end of load increment $\Delta\mbf{f}^{ext}$
\begin{equation}
  \mbf{f}^{int}(\mbf{r}+\Delta\mbf{r}) = \mbf{f}^{ext}+\Delta\mbf{f}^{ext}
  \label{eq:nonlinearequalibrium}
\end{equation}
By the linearization of the nodal force vector around known equilibrium state we can obtain
\begin{equation}
  \mbf{f}^{int}(\mbf{r})+\pard{\mbf{f}^{int}}{\mbf{r}}\Delta\mbf{r}+O(\|\Delta\mbf{r}\|^2)
\end{equation}
The derivative of internal force vector with respect to nodal displacements is called jacobian matrix and in solid mechanics as tangent stiffness matrix.
For the case of material non-linearity
\begin{eqnarray}
\mbf{f}^{e,int}(\mbf{r^e})&=&\int_{\Omega^e}\mbf{B}^T\mbf{\sigma}(\mbf{\eps}(\mbf{r^e}))\ d\Omega \\
\pard{\mbf{f}^{e,int}}{\mbf{d}^e} &=& \int_{\Omega^e}\mbf{B}^T\pard{\mbf{\sigma}}{\mbf{\eps}}\pard{\mbf{\eps}}{\mbf{r^e}}\ d\Omega =
\int_{\Omega^e}\mbf{B}^T\pard{\mbf{\sigma}}{\mbf{\eps}}\mbf{B}{\mbf{r}^e}\ d\Omega =
\int_{\Omega^e}\mbf{B}^T\mbf{D}\mbf{B}{\mbf{r}^e}\ d\Omega
\label{eq:newtonlinearization}
\end{eqnarray}

\subsection{Newton-Raphson method}
The method is based on splitting of the loading process into series of subsequent incremental loading steps in which the incremental loading vector $\Delta\mbf{f}$ is applied. We are looking for the equilibrium at the end of loading step~\ref{eq:nonlinearequalibrium} using the iterative procedure outlined in~\ref{eq:newtonlinearization}. The algorithm is graphically outlined in Fig.~\ref{fig:newtonraphson} for a system with one unknown and summarized in Table~\ref{tab:newtonraphson}.

\begin{figure}
  \begin{center}
    \includegraphics[width=0.8\textwidth]{newtonraphson.png}
  \end{center}
  \label{fig:newtonraphson}
  \caption{Illustration of Newton-Raphson method}
\end{figure}


\begin{table}[h!]
  \begin{center}
  \begin{tabular}{|l|l|}
    \hline
    Given \\
    \;\;$\mbf{f}^{ext}_{n-1}$\\
    \;\;$\mbf{f}^{ext}_{n}=\mbf{f}^{ext}_{n-1}+\delta\mbf{f}^{ext}_n$\\
    \;\;$\mbf{r}^{0}_{n} = \mbf{r}_{n-1}$\\
    Looking for $\mbf{r}_n$, such that $\mbf{f}^{int}(\mbf{r}_n) = \mbf{f}^{ext}_n$\\
    Solve for $i=1,2,\cdots$\\ 
    \;\;$\mbf{K}^{i}\delta\mbf{r}^{i} = \mbf{f}^{ext}_n-\mbf{f}^{int}_n(\mbf{r}^{i-1}_{n})$\\
    \;\;$\mbf{r}^i_{n} = \mbf{r}^{i-1}_{n}+\delta\mbf{r}^i$\\
    Until $\|\mbf{f}^{ext}_n-\mbf{f}^{int}_n(\mbf{r}^{i-1}_{n})\| \le \eps$\\
    \hline
  \end{tabular}
  \end{center}
  \caption{Newton-Raphson method}
  \label{tab:newtonraphson}
\end{table}

Based on update strategy for stiffness matrix, one can obtain different variants of the method. 
When the stiffness matrix $\mbf{K}^{i}$ is updated in each iteration, the full Newton-Raphson method is obtained. When stiffness matrix only every n-th iteration, one speaks about modified Newton-Raphson method. Finally, when the stiffness matrix is updated only at the beginning of the loading step, one obtains so called initial stiffness method.
For the full Newton-Raphson method a quadratic convergence is obtained.

One can implement two blends of Newton-Raphson algorithm, where the loading can be driven by load control or by displacement control, where the prescribed increments of displacements are applied to selected DOFs.

\subsection{Arc-length method}
We start by assuming the parametrized loading, in which the total external load vector is expressed as 
$$\mbf{f}^{ext}(\lambda) = +\lambda\mbf{f}^{ext}_p$$
where $\mbf{f}_p$ is proportional, reference load vector, and $\lambda$ is load scaling parameter, 
The arc-length method is based on idea of controlling the length passed along the loading path. For the differential length of loading  path we can write
\begin{equation}
  \Delta l = \sqrt{\Delta\mbf{r}^T\Delta\mbf{r}+(c^2\Delta\lambda^2\mbf{f}^T_p\mbf{f}_p)}
  \label{eq:arclength}
\end{equation}
where $c$ is coefficient of generalized metrics used to define $\Delta l$ (taking into account different units of displacement and load).
For selected increment of loading path length $\Delta l$, we are looking for the equilibrium, where the unknowns are nodal displacements $\mbf{r}$ and the load scaling parameter $\lambda$. We have the equilibrium equation and additional scalar equation~\ref{eq:arclength}:
\begin{eqnarray}
  \label{eq:alm1}
  \mbf{f}^{int}(\mbf{r}_n) &=& \mbf{f}^{ext}(\lambda_n\mbf{f}_p)\\
  \label{eq:alm2}
  \Delta l_n^2&=&\Delta\mbf{r}_n^T\Delta\mbf{r}_n+c^2\Delta\lambda^2\mbf{f}^T_p\mbf{f}_p
\end{eqnarray}

\begin{figure}
  \begin{center}
    \includegraphics[width=0.8\textwidth]{arclength.png}
  \end{center}
  \label{fig:arclength}
  \caption{Illustration of Acr-length method}
\end{figure}


At the end of n-th loading step and i-th iteration the displacement vector can be written as
$\mbf{r}^i_n = \mbf{r}^{i-1}_n+\delta\mbf{r}$ and similarly the load scaling parameter as $\lambda_n^i = \lambda_n^{i-1}+\delta\lambda$. Substituting this into equilibrium equation~\ref{eq:alm1} we get
$$\mbf{f}^{int}(\mbf{r}_n^{i-1}+\delta\mbf{r})=\mbf{f}^{ext}((\lambda_n^{i-1}+\delta\lambda)\mbf{f}_p)$$
By linearization of $\mbf{F}^{int}$ around known state $\mbf{r}_n^{i-1}$ we get
$$\mbf{f}^{int}_n(\mbf{r}_n^{i-1})+\mbf{K}_n^{i-1}\delta\mbf{r} = \mbf{f}^{ext,i-1}+\delta\lambda\mbf{f}_p$$
and finally for unknown $\delta\mbf{r}$
\begin{equation}
  \delta\mbf{r} = \underbrace{(\mbf{K}_n^{i-1})^{-1}(\mbf{f}^{ext,i-1}-\mbf{f}^{int}(\mbf{r}_n^{i-1}))}_{\delta\mbf{r}_r} + \delta\lambda\underbrace{(\mbf{K}_n^{i-1})^{-1}\mbf{f}_p}_{\delta\mbf{r}_\lambda}
\end{equation}
Note that the vectors $\delta\mbf{r}_r$ and $\delta\mbf{r}_\lambda$ can be computed and the only unknown remaining is the incremental change of loading parameter $\delta\lambda$, which could be determined from~\ref{eq:alm2}
\begin{eqnarray}
  \Delta l_n^2&=&(\Delta\mbf{r}_n^{i-1}+\delta\mbf{r}_r+\delta\lambda\delta\mbf{r}_\lambda)^T(\Delta\mbf{r}_n^{i-1}+\delta\mbf{r}_r+\delta\lambda\delta\mbf{r}_\lambda) + c^2(\Delta\lambda_n^{i-1}+\delta\lambda)^2\mbf{f}^T_p\mbf{f}_p
  \label{eq:quadraticlambda}
\end{eqnarray}
This finally yields a quadratic equation for unknown increment of loading parameter $\delta\lambda$. The algorithm is summarized in Table~\ref{tab:alm}.

\begin{table}[h!]
  \begin{center}
  \begin{tabular}{|l|l|}
    \hline
    Given \\
    \;\;$\mbf{f}^{ext}_{n-1}, \mbf{f}_p$\\
    \;\;$\mbf{r}^{0}_{n} = \mbf{r}_{n-1}$\\
    Evaluate\\
    $\;\;\delta\mbf{r}_\lambda = (\mbf{K}_n)^{-1}\mbf{f}_p$\\
    $\;\;\Delta\lambda^0=\pm{\Delta l}/{\sqrt{\delta\mbf{r}_\lambda^T\delta\mbf{r}_\lambda+c^2\mbf{f}_p^T\mbf{f}_p}}$\\
    $\;\;\Delta\mbf{r}_n^0 = (\mbf{K}_n)^{-1}(\Delta\lambda\mbf{f}_p) = (\mbf{K}_n)^{-1}((\lambda_n+\Delta\lambda^0)\mbf{f}_p - \mbf{f}^{int,0}_n)$\\
    Repeat for $i=1,2,\cdots$\\ 
    $\;\;\delta\mbf{r}_\lambda = (\mbf{K}_n^{i-1})^{-1}\mbf{f}_p$\\
    $\;\;\delta\mbf{r}_r = (\mbf{K}_n^{i-1})^{-1}(\lambda_n^{i-1}\mbf{f}_p-\mbf{f}^{int}(\mbf{r}_n^{i-1}))$\\
    $\;\;$Solve quadratic equation~\ref{eq:quadraticlambda} for $\delta\lambda$\\
    $\;\;\delta\mbf{r}^i = \delta\mbf{r}_r+\delta\lambda\delta\mbf{r}_\lambda$\\
    $\;\;\Delta\mbf{r}_n^i = \Delta\mbf{r}^{i-1}+\delta\mbf{r}^i,\ \mbf{r}_n^i = \mbf{r}_n^{i-1}+\delta\mbf{r}^i$\\
    $\;\;\lambda^i = \lambda^{i-1}+\delta\lambda,\ \Delta\Lambda_n^i = \Delta\Lambda_n^{i-1}+\delta\lambda$\\
    Until convergence reached\\
    \hline
  \end{tabular}
  \end{center}
  \caption{Newton-Raphson method}
  \label{tab:newtonraphson}
\end{table}












\section{Non-stationary linear transport model}
\label{NonLinTrans}
The weak form of diffusion-type differential equation leads to
\begin{eqnarray}
\mbf{K} \mbf{r} + \mbf{C}\frac{{\rm d}\mbf{r}}{{\rm d} t} = \mbf{F}\label{NonLinTrans:1},
\end{eqnarray}
where the matrix $\mbf{K}$ is a general non-symmetric conductivity matrix, $\mbf{C}$ is a general capacity matrix and the vector $\mbf{F}$ contains contributions from external and internal sources. The vector of unknowns, $\mbf{r}$, can hold nodal values of temperature, humidity, or concentration fields, for example.

Time discretization is based on a generalized trapezoidal rule. Let us assume that the solution is known at time $t$ and the time increment is $\Delta t$. The parameter $\alpha\in\langle 0, 1\rangle$ defines a type of integration scheme; $\alpha=0$ results in an explicit (forward) method, $\alpha=0.5$ refers to the Crank-Nicolson method, and $\alpha=1$ means an implicit (backward) method. The appromation of solution vector and its time derivative yield
\begin{eqnarray}
\tau &=& t+\alpha\Delta t = (t+\Delta t) - (1-\alpha)\Delta t,\label{NonLinTrans:2}\\
\mbf{r}_{\tau} &=& (1-\alpha)\mbf{r}_t+\alpha\mbf{r}_{t+\Delta t},\label{NonLinTrans:3}\\
\frac{{\rm d}\mbf{r}}{{\rm d} t} &=& \frac{1}{\Delta t}\left(\mbf{r}_{t+\Delta t}-\mbf{r}_t\right).\label{NonLinTrans:4}\\
\mbf{F}_{\tau} &=& (1-\alpha)\mbf{F}_t+\alpha\mbf{F}_{t+\Delta t},\label{NonLinTrans:5}
\end{eqnarray}

Let us assume that \refeq{NonLinTrans:1} should be satisfied at time $\tau$. Inserting \refeqsr{NonLinTrans:3}{NonLinTrans:5}
into \refeq{NonLinTrans:1} leads to
\begin{eqnarray}
\left[\alpha \mbf{K} + \frac{1}{\Delta t} \mbf{C} \right] \mbf{r}_{t+\Delta t} = 
\left[(\alpha-1) \mbf{K} + \frac{1}{\Delta t} \mbf{C} \right] \mbf{r}_{t} +
(1-\alpha) \mbf{F}_{t} + \alpha \mbf{F}_{t+\Delta t}\label{NonLinTrans:6}
\end{eqnarray}
where the conductivity matrix $\mbf{K}$ contains also a contribution from convection, since it depends on
$\mbf{r}_{t+\Delta t}$
\begin{eqnarray}
\mbf{K} = \int_{\Omega}\mbf{B}^T \lambda \mbf{B} \ud\Omega + \underbrace{\int_{{\Gamma}_{\overline{c}}} \mbf{N}^T h \mbf{N} \ud \Gamma}_{\textrm{Convection}}\label{NonLinTrans:7}
\end{eqnarray}
The vectors $\mbf{F}_t$ or $\mbf{F}_{t+\Delta t}$ contain all known contributions
\begin{eqnarray}
\mbf{F}_t = - \underbrace{\int_{\Gamma_{\overline{q}}} \mbf{N}^T \overline{q}_t \ud \Gamma}_{\textrm{Given flow}} +
\underbrace{\int_{\Gamma_{\overline{c}}} \mbf{N}^T h T_{\infty,t} \ud \Gamma}_{\textrm{Convection}} +
\underbrace{\int_{\Omega}\mbf{N}^T\overline{Q}_t\ud\Omega}_{\textrm{Source}}\label{NonLinTrans:8}
\end{eqnarray}

\section{Non-stationary nonlinear transport model}
\label{NonNonTrans}
In a nonlinear model, \refeq{NonLinTrans:1} is modified to
\begin{eqnarray}
\mbf{K}(\mbf{r}) \mbf{r} + \mbf{C}(\mbf{r})\frac{{\rm d}\mbf{r}}{{\rm d} t} = \mbf{F}(\mbf{r})\label{NonNonTrans:1},
\end{eqnarray}

Time discretization is the same as in \refeqsr{NonLinTrans:2}{NonLinTrans:4} but the assumption in \refeq{NonLinTrans:8} is not true anymore. Let us assume that \refeq{NonNonTrans:1} should be satisfied at time $\tau\in\langle t,t+\Delta t \rangle$. By substituting of \refeqsr{NonLinTrans:3}{NonLinTrans:4} into \refeq{NonNonTrans:1} leads to the following equation
\begin{equation}
\left[(1-\alpha)\mbf{r}_t + \alpha\mbf{r}_{t+\Delta t} \right] \mbf{K}_{\tau}(\mbf{r}_\tau) +
\left[\del{\mbf{r}_{t+\Delta t}-\mbf{r}_t}{\Delta t}\right] \mbf{C}_{\tau}(\mbf{r}_\tau) = 
\mbf{F}_{\tau}(\mbf{r}_\tau).\label{NonNonTrans:5}
\end{equation}

\refeq{NonNonTrans:5} is non-linear and the Newton method is used to obtain the solution. First, the \refeq{NonNonTrans:5} is 
transformed into a residual form with the residuum vector $\mbf{R}_{\tau}$, which should converge to the zero vector
\begin{equation}
\mbf{R}_{\tau} = 
\left[(1-\alpha)\mbf{r}_t + \alpha\mbf{r}_{t+\Delta t} \right] \mbf{K}_{\tau}(\mbf{r}_\tau) +
\left[\del{\mbf{r}_{t+\Delta t}-\mbf{r}_t}{\Delta t}\right] \mbf{C}_{\tau}(\mbf{r}_\tau) -
\mbf{F}_{\tau}(\mbf{r}_\tau) \to \mbf{0}.\label{NonNonTrans:6}
\end{equation}

A new residual vector at the next iteration, $\mbf{R}_\tau^{i+1}$, can determined from the previous residual vector, $\mbf{R}_{\tau}^i$, and its derivative simply by linearization. Since the aim is getting an increment of solution vector, $\Delta\mbf{r}_{\tau}^i$, the new residual vector $\mbf{R}_{\tau}^{i+1}$ is set to zero
\begin{eqnarray}
\mbf{R}_\tau^{i+1} &\approx& \mbf{R}_{\tau}^i+\frac{\partial{\mbf{R}_{\tau}^i}}{\partial\mbf{r}_t} \Delta\mbf{r}_{\tau}^i = \mbf{0},\label{NonNonTrans:7}\\
\Delta \mbf{r}_{\tau}^i &=& - \left[\frac{\partial{\mbf{R}_{\tau}^i}}{\partial\mbf{r}_t}\right]^{-1} \mbf{R}_{\tau}^i\label{NonNonTrans:8}.
\end{eqnarray}
Deriving \refeq{NonNonTrans:6} and inserting to \refeq{NonNonTrans:8} leads to
\begin{eqnarray}
\mbf{\tilde K}_\tau^i &=& \frac{\partial{\mbf{R}_{\tau}^i}}{\partial\mbf{r}_t} = -\alpha \mbf{K}_{\tau}^i(\mbf{r}) - \del{1}{\Delta
t}\mbf{C}_{\tau}^i(\mbf{r})\label{NonNonTrans:9},\\
\Delta \mbf{r}_{\tau}^i &=& - \left[\mbf{\tilde K}_\tau^i\right]^{-1} \mbf{R}_{\tau}^i,\label{NonNonTrans:10}
\end{eqnarray}
which gives the resulting increment of the solution vector $\Delta \mbf{r}_{\tau}^i$
\begin{equation}
\begin{split}
\Delta \mbf{r}_{\tau}^i = - \left[\mbf{\tilde K}_\tau^i\right]^{-1} \Big\{ &\left[(1-\alpha)\mbf{r}_t + \alpha\mbf{r}_{t+\Delta t} \right] \mbf{K}_{\tau}^i(\mbf{r}_\tau) +\\
&\left[\del{\mbf{r}_{t+\Delta t}-\mbf{r}_t}{\Delta t}\right] \mbf{C}_{\tau}^i(\mbf{r}_\tau) - \mbf{F}_{\tau}(\mbf{r}_\tau) \Big\},\label{NonNonTrans:11}
\end{split}
\end{equation}
and the new total solution vector at time $t + \Delta t$ is obtained in each iteration
\begin{eqnarray}
\mbf{r}_{t+\Delta t}^{i+1}=\mbf{r}_{t+\Delta t}^{i} + \Delta\mbf{r}_{\tau}^i.\label{NonNonTrans:12}
\end{eqnarray}

There are two options how to initialize the solution vector at time $t + \Delta t$. While the first case applies linearization with a known derivative, the second case simply starts from the previous solution vector. The second method in \refeq{NonNonTrans:14} is implemented in OOFEM.
\begin{eqnarray}
\mbf{r}_{t+\Delta t}^{0} = \mbf{r}_t + \Delta t\frac{\partial{\mbf{r}_t}}{\partial t},\label{NonNonTrans:13}\\
\mbf{r}_{t+\Delta t}^{0} = \mbf{r}_t.\label{NonNonTrans:14}
\end{eqnarray}

Note that the matrices $\mbf{K}(\mbf{r}_\tau), \mbf{C}(\mbf{r}_\tau)$ and the vector $\mbf{F}(\mbf{r}_\tau)$ depend on the solution vector $\mbf{r}_\tau$. For this reason, the matrices are updated in each iteration step (Newton method) or only after several steps (modified Newton method). The residuum $\mbf{R}_\tau^{i}$ and the vector $\mbf{F}_\tau(\mbf{r}_\tau)$ are updated in each iteration, using the most recent capacity and conductivity matrices.

\subsection{Heat flux from radiation}

Heat flow from a body surrounded by a medium at a temperature $T_\infty$ is governed by the Stefan-Boltzmann Law
\begin{eqnarray}
q(T, T_\infty) = \varepsilon \sigma (T^4 - T_\infty^4)\label{eq:StefanBoltzmann}
\end{eqnarray}
where $\varepsilon\in\langle 0, 1 \rangle$ represents emissivity between the surface and the boundary at temperature $T_\infty$. $\sigma=5.67\cdot 10^{-8}$ W/m$^{-2}$K$^{-4}$ stands for a Stefan-Boltzmann constant. Transport elements in OOFEM implement \refeq{eq:StefanBoltzmann} and require non-linear solver.

Alternatively (not implemented), a linearization using Taylor expansion around $T_\infty$ and neglecting higher-order terms results to
\begin{eqnarray}
q(T, T_\infty) &\approx& q(T=T_\infty) + \frac{\partial q(T,T_\infty)}{\partial T_\infty} (T_\infty-T) = 4\varepsilon \sigma T_\infty^3 (T-T_\infty)
\end{eqnarray}
leading to so-called radiation heat transfer coefficient $\alpha_{rad}=4\varepsilon \sigma T_\infty^3$. The latter resembles similar coefficient as in convective heat transfer. Other methods for \refeq{eq:StefanBoltzmann} could be based on Oseen or Newton-Kantorovich linearization. Also, radiative heat transfer coefficient $\alpha_{rad}$ can be expressed as \cite[pp.28]{Baehr:06}
\begin{eqnarray}
q(T, T_\infty) = \varepsilon \sigma \frac{T^4 - T_\infty^4}{T-T_\infty}(T-T_\infty) = \underbrace{\varepsilon \sigma (T^2+T_\infty^2)(T+T_\infty)}_{\alpha_{rad}}(T-T_\infty)
\end{eqnarray}


\end{document}